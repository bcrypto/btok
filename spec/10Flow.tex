\chapter{Выпуск билета аутентификации}\label{FLOW}

\section{Схема}\label{FLOW.Common}

В настоящем разделе определяется схема взаимодействия ПС, КП, КТ и его
владельца, СИ и его терминала для аутентификации КТ с выдачей ПС билета
аутентификации.
%
Аутентификация и выпуск билета проводятся СИ и его терминалом по запросу ПС при 
помощи КП и с согласия владельца.

Утверждения, которые приводятся в билете, могут быть известны СИ
(например, утверждения аутентификации), либо могут храниться на КТ в виде 
идентификационных атрибутов, и тогда терминалу требуется их получить.
%
Перед передачей утверждений пользователь дает согласие на передачу, а КТ 
проверяет, что запрошенные атрибуты подкреплены правами в сертификате 
терминала. Атрибуты пересылаются по защищенному каналу.

Непосредственно КТ аутентифицирует терминал.
Для аутентификации используется протокол BAUTH (односторонний)
и этому протоколу предшествует протокол BPACE.

Определяемая схема может использоваться для построения систем массовой 
аутентификации. \doubt{Схема может адаптироваться к нуждам систем в части 
организации пересылок, выполнения дополнительных действий. 
При адаптации должна сохраняться базовая криптографическая логика.}

\section{Объекты}

В схеме используются следующие объекты:
\begin{itemize}
\item[--]
$\Req_\text{ПС}$~--- запрос аутентификации;
\item[--]
$\Chart_\text{ПС}$~--- перечень утверждений, запрашиваемых ПС;
\item[--]
$\Chart_\text{КП}$~--- перечень идентификационных атрибутов, на предоставление 
которых дает согласие владелец КТ с помощью КП. 
%
КП формирует перечень атрибутов $\Chart_\text{КП}$, на предоставление которых ПС 
владелец КТ дает согласие. Перечень $\Chart_\text{КП}$ является подмножеством 
$\Chart_\text{ПС}$ и не включает не одобренные владельцем необязательные 
атрибуты из $\Chart_\text{ПС}$.  
%
Перечень кодируется словом 
прав доступа к прикладной  программе eID. Это слово, в свою очередь, 
представляется объектом типа \texttt{CertHAT} (см.~\ref{DATA.Access});
\item[--]
$\Cert(Id_T,Q_T)$~--- сертификат терминала. Содержит слово прав доступа
к прикладной программе eID;
\item[--]
$\Attr_\text{КТ}$~--- идентификационных атрибуты (группы данных и 
дополнительные атрибуты), которые хранятся на КТ. 
\end{itemize}

\section{Сообщения}

Стороны обмениваются сообщениями вспомогательных протоколов BPACE и BAUTH. 
Сообщения этих протоколов маркируются индексом, в котором указывается 
название протокола. Например, $\text{M1}_\text{BAUTH}$~--- 
это сообщение M1 протокола BAUTH. 

Стороны могут обмениваться дополнительными информационными сообщениями, 
например, характеристиками КТ или параметрами дополнительных идентификационных 
атрибутов.  

Сообщения протоколов BPACE и BAUTH и информационные сообщения высылаются 
между следующими основными сообщениями: 

M1 ($\text{ПС}\to\text{КП}$): 
$\Req_\text{ПС}=\langle\langle\Chart_\text{ПС}\rangle\rangle$;

M2 ($\text{КП}\to\text{СИ}$): 
$\langle\langle\Req_\text{ПС},\Chart_\text{КП}\rangle\rangle$;

M3 ($\text{СИ}\to\text{КП}$): 
$\langle\langle\Cert(Id_\text{Т},Q_\text{Т})\rangle\rangle$;

M4 ($\text{КП}\to\text{Т}$): 
$\langle\langle\text{M1}_\text{BAUTH}\rangle\rangle$;

M5 ($\text{Т}\to\text{КТ}$): 
$\langle\langle\Chart_\text{КП}\rangle\rangle$;

M6 ($\text{КТ}\to\text{Т}$): 
$\langle\langle\Attr_\text{КТ}\rangle\rangle$.

Сообщения M3, M4 фактически являются сообщениями BAUTH. Они выделены,
чтобы показать, как происходит переключение с соединения КП -- СИ на соединение 
КТ -- теримнал.

Сообщения протокола BAUTH пересылаются по защищенному соединению, 
которое устанавливается после выполнения протокола BPACE. 
Сообщения M5, M6 пересылаются по защищенному соединению, которое 
устанавливается после выполнения протокола BAUTH. Хотя КП является 
посредником при передаче сообщений, он не может раскрыть их содержимое. 

В сообщении M2 КП может дополнительно передать терминалу информацию о КТ, 
например, списки установленных на КТ приложений и корневых сертификатов.
Перечень дополнительной информации о КТ и формат ее представления в M3 
определяются за рамками настоящего стандарта. 
Формат сообщений детализируется в~\ref{CMDS}.

\section{Шаги}


В результате выполнения протокола ПC получает подписанный терминалом билет 
аутентификации (в том числе атрибуты владельца КT и атрибуты 
аутентификации), либо сообщение об ошибке. Возврат сообщения об ошибке 
означает либо сбой при передаче сообщений протокола, либо нарушение 
целостности сообщений, либо нарушение их подлинности, либо ошибку 
аутентификации некоторой стороны протокола. 

\subsection{Шаг 1: отправка запроса аутентификации}

ПС отправляет КП запрос аутентификации. 
Этот запрос содержит перечень утверждений, которые ПС просит 
предоставить. Запрос обозначается через $\Req_\text{ПС}$,
перечень~--- через $\Chart_\text{ПС}$, 
собственно утверждения~--- через~$\Attr_\text{КТ}$.

Запрос должен быть волатильным: запросы с одинаковым содержимым должны 
отличаться. Например, запрос может включать отметку времени или уникальную
синхропосылку.

Может требоваться, чтобы ПС подписывала запрос или включать в него аттестаты,
подтверждающие права доступа к запрошенным атрибутам. В таких случаях
проверка запроса включает проверку подписи и прав.

\subsection{Шаг 2: обработка запроса аутентификации}

КП проверяет запрос аутентификации, выделяет в нем перечень $\Chart_\text{ПС}$.
%
КП согласует перечень с владельцем КТ.
%
Владелец выбирает устраивающие его пункты перечня, в результате чего 
формируется перечень~$\Chart_\text{КП}$. Кроме пунктов, запрошенных ПС,
перечень может включать технические пункты, необходимые для взаимодействия с 
СИ. 

Окончательный перечень оформляется в виде слов прав доступа к прикладной 
программе \texttt{eID}. Эти права представляются 
объектом типа \texttt{CertHAT}, определенным в~\ref{DATA.Access}.

КП вычисляет хэш-значение $H_\text{КП}=\texttt{belt-hash}(\Req_\text{ПС})$ 
и отправляет запрос СИ вместе с $\Chart_\text{КП}$. СИ может быть заранее
известен, либо КП может выбрать СИ из списка с участием владельца КП. 
Спиоск подходящих СИ может быть дан в~$\Req_\texttt{ПС}$.

\subsection{Шаг 3: парольная аутентификация}

КП и КТ выполняют протокол BPACE (обмениваясь сообщениями 
$\text{M1}_\text{BPACE}$~---  $\text{M4}_\text{BPACE}$) и формируют общий ключ 
$K_0$. Стороны используют $P$ в качестве пароля и $\Chart_\text{КП}$ в качестве 
приветственного сообщения $\hello_\text{КП}$. КТ должен сохранить 
$\Chart_\text{КП}$ для дальнейшего использования.

КП и КТ создают защищенное соединение на ключе $K_0$.

\subsection{Шаг 4: обработка запроса аутентификации}

СИ получает запрос $\Req_\text{ПС}$ и перечень $\Chart_\text{КП}$.

СИ проверяет присланный запрос, выделяет в нем перечень 
$\Chart_\text{ПС}$ и проверяет его соответствие $\Chart_\text{КП}$. 
%
КП вычисляет хэш-значение $H_\text{СИ}=\texttt{belt-hash}(\Req_\text{ПС})$.

СИ выбирает терминал, права доступа которого позволяют запрашивать все 
атрибуты владельца КТ в соответствии с перечнем $\Chart_\text{КП}$

СИ отправляет КП сертификат $\Cert(Id_T,Q_T)$ выбранного терминала.

\subsection{Шаг 5: открытие BAUTH}

КП получает от СИ сертификат $\Cert(Id_T,Q_T)$.

КП начинает выполнение протокола BAUTH от имени Т, отправляя КТ сертификат 
$\Cert(Id_\text{Т}, Q_\text{Т})$ как часть сообщения $\text{M0}_\text{BAUTH}$.

КТ:
\begin{enumerate}
\item
начинает выполнение протокола BAUTH, получая от КП сертификат 
$\Cert(Id_\text{Т},Q_\text{Т})$ как часть сообщения $\text{M0}_\text{BAUTH}$; 
\item
проверяет $\Cert(Id_\text{Т}, Q_\text{Т})$;
\item
выделяет в $\Cert(Id_\text{Т}, Q_\text{Т})$ права доступа Т к атрибутам владельца и 
проверяет, что Т имеет права доступа ко всем атрибутам из перечня 
$\Chart_\text{КП}$, полученного на \doubt{шаге 2}. 
\end{enumerate}

КП отправляет приветственное сообщение~$\hello_\text{Т}$ 
как часть сообщения $\text{M0}_\text{BAUTH}$. 
В качестве $\hello_\text{Т}$ используется хэш-значение $H_\text{КП}$. 

КТ:
\begin{enumerate}
\item
продолжает выполнение протокола BAUTH, получая от КП приветственное 
сообщение $\hello_\text{Т}$ как часть сообщения $\text{M0}_\text{BAUTH}$; 
\item
формирует и отправляет КП сообщение $\text{M1}_\text{BAUTH}$ с пустым 
приветственным сообщением $\hello_\text{КТ}$. 
\end{enumerate}

Обмен сообщениями между КТ и КП идет по защищенному соединению,
открытому на \doubt{шаге 4}.

КП формирует и отправляет Т сообщение $\text{M1}_\text{BAUTH}$.

\subsection{Шаг 6: завершение BAUTH}

Т получает сообщение $\text{M1}_\text{BAUTH}$.

Т и КТ:
\begin{enumerate}
\item
завершают выполнение протокола BAUTH (обмениваясь сообщениями $\text{M2}_\text{BAUTH}$ и 
$\text{M3}_\text{BAUTH}$) и формируют общий ключ $K_0$. Протокол выполняется при 
посредничестве КП, которая передает на КТ команды Т и возвращает обратно 
соответствующие ответы;  
\item
создают защищенное соединение на ключе $K_0$. При этом КТ переключается
на новое соединение с предыдущего защищенного соединения с КП;
\item
Т отправляет, а КТ получает фрагменты сообщения M3. Каждый фрагмент 
содержит атрибут из перечня $\Chart_\text{КП}$. КТ проверяет, что присланный атрибут 
содержатся в перечне, полученном на шаге 3.1; 
\item
КТ отправляет, а Т получает фрагменты сообщения M4. Каждый фрагмент 
содержит значение запрошенного атрибута из перечня $\Chart_\text{КП}$; 
\end{enumerate}

\subsection{Шаг 7: выпуск билета аутентификации}

Т:
\begin{enumerate}
\item
собирает фрагменты сообщение M4;
\item
определяет по M4 атрибуты $\Attr_\text{КТ}$;
\item
определяет атрибуты $\Attr_\text{Т}$;
\item
формирует билет аутентификации.
\end{enumerate}

\begin{note}
Примечание 1~-- 
Если $\Cert(Id_\text{Т}, Q_\text{Т})$ представляет собой маршрут 
сертификации, то КП на шаге 5.1 передает последовательные 
(от \doubt{корневого} к конечному) сертификаты маршрута. Правила передачи 
маршрута сертификации описаны в~\ref{CERTS.Path}. 
\end{note}

\begin{note}
Примечание 2~-- 
Защищенное соединение между КТ и терминалом может поддерживаться 
определенное время после аутентификации, например для обмена 
дополнительными информационными сообщениями. 
\end{note}

\begin{note}
Примечание 3~-- 
При возникновении ошибки во время выполнения протокола терминал 
должен фиксировать информацию об ошибке в компоненте status билета 
аутентификации. КП должна, по возможности, сообщать терминалу об ошибках во время 
выполнения протокола. Возможны ситуации, когда сервер не получает 
информацию об ошибке, либо не может доставить КП билет с ее описанием. 
Такими случаями являются ошибки на первом шаге протокола, разрыв 
соединения и т. п.  
\end{note}

\begin{note}
Примечание 4~-- 
Отсутствие запрашиваемого атрибута владельца КТ не является 
ошибкой, если этот атрибут не является обязательным. При отсутствии 
необязательного атрибута терминал может пометить его в билете как <<отсутствующий 
на КТ>>. 
\end{note}

