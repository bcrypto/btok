\clearpage
\chapter*{\mbox{}\hfill Поправка к официальной редакции\hfill\mbox{}}

\mbox{}

{\small
\begin{longtable}{|p{3.0cm}|p{6.0cm}|p{6.6cm}|}
\hline
В каком месте & Напечатано & Должно быть\\
\hline
\hline
Пункт~\ref{CRYPTO.SM.Algs.Encr},\par шаг~3 
&
$T\gets\texttt{belt-mac}(\llangle I, Y\rrangle, K_1)$.
&
$T\gets\texttt{belt-mac}(S\parallel \llangle I, Y\rrangle, K_1)$.
\\
\hline
Пункт~\ref{CRYPTO.SM.Algs.Decr},\par шаг~3 
&
Проверить, что\par 
$T=\texttt{belt-mac}(\llangle I, Y, T\rrangle, K_1)$. 
&
Проверить, что\par 
$T=\texttt{belt-mac}(S\parallel \llangle I, Y\rrangle, K_1)$.
\\
\hline
Подраздел~\ref{FLOW.Msgs},\par таблица~\ref{Table.FLOW.Msgs},
шаг 3,\par сообщение
&
$\llangle\text{M0}_\text{BPACE},\text{M1}_\text{BPACE}\rrangle$
&
$\llangle\text{M1}_\text{BPACE}\rrangle$
\\
\hline
Подраздел~\ref{FLOW.Msgs},\par таблица~\ref{Table.FLOW.Msgs},
шаг 3,\par примечание
&
$\text{M0}_\text{BPACE}=\Chart_\text{В}$
&
$\hello_\text{КП}=\llangle\Chart_\text{В}\rrangle$
\\
\hline
Подраздел~\ref{FLOW.Msgs},\par таблица~\ref{Table.FLOW.Msgs},
шаг 5,\par примечание
&
$\text{M0}_\text{BAUTH}=(H_\text{КП},\Cert(Id_\text{Т},Q_\text{Т}))$
&
$\text{M0}_\text{BAUTH}=
\llangle\hello_\text{Т},\Cert(Id_\text{Т},Q_\text{Т})\rrangle$ 
\\
\hline
Подраздел~\ref{FLOW.Steps},\par шаг~3
&
В качестве приветственного сообщения $\hello_\text{КП}$ протокола BPACE
используется перечень~$\Chart_\text{КП}$. КТ должен сохранить 
$\Chart_\text{КП}$ для дальнейшего использования.
&
В качестве приветственного сообщения $\hello_\text{КП}$ протокола BPACE
используется перечень~$\Chart_\text{В}$ и возможно другие данные. 
КТ должен сохранить $\Chart_\text{В}$ для дальнейшего использования.
\\
\hline
Подраздел~\ref{FLOW.Steps},\par шаг~5
&
В качестве $\hello_\text{Т}$ используется хэш-значение $H_\text{СИ}$. 
&
В $\hello_\text{Т}$ указывается хэш-значение $H_\text{СИ}$ ($H_\text{КП}$
на стороне КП), перечень~$\Chart_\text{В}$ и возможно другие данные.
\\
\hline
Пункт~\ref{Oper.Descr.SetBAUTH},\par абзац 2
&
В компоненте CDF все объекты данных являются обязательными, 
они должны передаваться с помощью одной команды, 
т.~е. использование цепочки команд <MSE: Set AT> не допускается. 
При этом порядок следования объектов данных в компоненте CDF не важен. 
&
В компоненте CDF обязательным является только первый объект.
Объекты должны передаваться с помощью одной команды, т.~е. использование
цепочки команд <MSE: Set AT> не допускается.
\\
\hline
Пункт~\ref{Oper.Descr.SetBAUTH},\par абзац 3 
&
Команда может вызываться при выборе мастер-файла в состоянии PS 
непосредственно после выполнения протокола BPACE 
(см.~\ref{Oper.Descr.GABPACE}). 
&
Команда может вызываться при выборе мастер-файла в состоянии PS 
непосредственно после выполнения протокола BPACE 
(см.~\ref{Oper.Descr.GABPACE}). 
%
Приветственное сообщение $\hello_\text{Т}$ протокола BAUTH определяется как 
$\text{CDF}\parallel\texttt{CertHAT}^*$. Здесь~$\texttt{CertHAT}^*$~--- 
список объектов $\texttt{CertHAT}$, указанный ранее в команде инициализации 
протокола BPACE (см.~\ref{Oper.Descr.SetBPACE}). 
%
Приветственное сообщение $\hello_\text{КТ}$ протокола BAUTH полагается пустым.
\\
\hline
Пункт~\ref{Oper.Descr.SetBPACE}
&
Объекты типа \verb|CertHAT|, передаваемые в компоненте CDF, 
должны использоваться в протоколе BPACE как приветственное сообщение КП
(см.~\ref{CRYPTO.BPACE}).
%
Если передается несколько таких значений, то при формировании приветственного
сообщения они должны объединяться в порядке следования в компоненте CDF. Если же
не передается ни одного значения, то в качестве приветственного сообщения должно
использоваться пустое слово.
&
Компонент CDF должен использоваться в протоколе BPACE в качестве 
приветственного сообщения $\hello_\text{КП}$ (см.~\ref{CRYPTO.BPACE}).
\\
\hline
Пункт~\ref{CMDS.SM.EncrCmd},\par шаг~4 
&
$I\parallel\text{Le*}\parallel\text{CDF*}\parallel\hex{00}$
&
$I\parallel\text{Lс*}\parallel\text{CDF*}\parallel\hex{00}$
\\
\hline
\end{longtable}
}